%JAP or Advances in Applied Probability
%Combinatorics, Probability and Computing
%Theory of Probability and its application


% Motivation/Introduction to k-hop: device-to-device and/or station-to-device communications in cellular networks, 


\documentclass[12pt]{article}

\usepackage{amsfonts}
%\usepackage{newtxmath}
\usepackage{bm}
\usepackage{enumerate} 
\usepackage{amssymb,amsmath}
\usepackage{caption}
\usepackage{subcaption}
\usepackage{accents} 

\usepackage{dsfont}

\usepackage{stackengine}

\usepackage{accents} 

\let\Horig\H

\usepackage{tikz}
\usetikzlibrary{automata,topaths}
\usetikzlibrary{shapes}
\usetikzlibrary{plotmarks}
 
% \usepackage{wasysym} 
\usepackage{float} 

\usepackage{color} 
\definecolor{lightblue}{rgb}{0,0.2,0.5}
\usepackage[colorlinks=true, urlcolor=lightblue,linkcolor=lightblue, citecolor=lightblue]{hyperref}

\DeclareMathAlphabet{\eufrak}{U}{}{}{} 
\SetMathAlphabet\eufrak{normal}{U}{euf}{m}{n}
\SetMathAlphabet\eufrak{bold}{U}{euf}{b}{n}


% \usepackage{eucal}

% \usepackage{epsfig,latexsym}

% \usepackage{graphicx,amsmath,amssymb,latexsym,psfrag}

% \usepackage[polish]{babel}

% \usepackage{graphics,graphicx,amsmath}

\oddsidemargin=0cm \textwidth=16.5cm \textheight=23cm
\topmargin=-1.5cm
\newcommand{\R}{\mathbb{R}}
\newcommand{\V}{\mathbb{V}}
\newcommand{\T}{\mathbb{T}}
\newcommand{\C}{\mathbb{C}}
\newcommand{\E}{\mathbb{E}}
\newcommand{\IP}{\mathbb{P}}
%\newcommand{\bone}{\bone}
\newcommand{\bone}{{\bf 1}}
% \newcommand{\E}{\mathrm{E}}

\newcommand{\supp}{\mathrm{supp}}

%% \renewcommand{\le}{\leqslant}
%% \renewcommand{\leq}{\leqslant}

%% \renewcommand{\ge}{\geqslant}
%% \renewcommand{\geq}{\geqslant}

\newcommand{\conv}{\mathrm{conv}}
\newcommand{\card}{\mathrm{card}}
\newcommand{\grad}{\mathrm{grad}}
\newcommand{\N}{\mathbb{N}}
\newcommand{\bbf}{{\mathbf{f}}}
\newcommand{\bT}{\mathbb{T}}
\newcommand{\TP}{\widetilde{P}}
\newcommand{\tgi}{t\rightarrow \infty}
\newcommand{\ngi}{n\rightarrow \infty}
\newcommand{\algi}{\alpha \rightarrow \infty}
\newcommand{\xgi}{x\rightarrow \infty}
\newcommand{\oJ}{\overline{J}}
\newcommand{\og}{\overline{\gamma}}
\newcommand{\oL}{\overline{\Lambda}}
\newcommand{\EPSI}{\varepsilon}

\newcommand{\disc}{\mathrm{disc}}
\newcommand{\bZ}{\bold{Z}}
\newcommand{\bz}{\bold{z}}
\newcommand{\dtv}{{d_{\rm TV}}}
\newcommand{\dk}{{d_{\rm K}}}
\newcommand{\dw}{{d_{\rm W}}}

\newtheorem{prop}{Proposition}[section]
\newtheorem{assumption}[prop]{Assumption}
\newtheorem{lemma}[prop]{Lemma}
\newtheorem{definition}[prop]{Definition}
\newtheorem{corollary}[prop]{Corollary}
\newtheorem{thm}[prop]{Theorem}
\newtheorem{remark}[prop]{Remark}
\newtheorem{example}[prop]{Example}

% To define \widebar
\makeatletter
\newcommand*\rel@kern[1]{\kern#1\dimexpr\macc@kerna}
\newcommand*\widebar[1]{%
  \begingroup
  \def\mathaccent##1##2{%
    \rel@kern{0.8}%
    \overline{\rel@kern{-0.8}\macc@nucleus\rel@kern{0.2}}%
    \rel@kern{-0.2}%
  }%
  \macc@depth\@ne
  \let\math@bgroup\@empty \let\math@egroup\macc@set@skewchar
  \mathsurround\z@ \frozen@everymath{\mathgroup\macc@group\relax}%
  \macc@set@skewchar\relax
  \let\mathaccentV\macc@nested@a
  \macc@nested@a\relax111{#1}%
  \endgroup
}
\makeatother

\makeatletter
\DeclareRobustCommand\widecheck[1]{{\mathpalette\@widecheck{#1}}}
\def\@widecheck#1#2{%
    \setbox\z@\hbox{\m@th$#1#2$}%
    \setbox\tw@\hbox{\m@th$#1%
       \widehat{%
          \vrule\@width\z@\@height\ht\z@
          \vrule\@height\z@\@width\wd\z@}$}%
    \dp\tw@-\ht\z@
    \@tempdima\ht\z@ \advance\@tempdima2\ht\tw@ \divide\@tempdima\thr@@
    \setbox\tw@\hbox{%
       \raise\@tempdima\hbox{\scalebox{1}[-1]{\lower\@tempdima\box
\tw@}}}%
    {\ooalign{\box\tw@ \cr \box\z@}}}
\makeatother

%\def\bone{\vmathbb{1}}
\def\bp{\noindent{\it Proof.}\ }
\def\ep{\hfill $\Box$}
\newcommand{\bt}{\mathbf{t}}
\def\({\left(}
\def\){\right)}
% \theoremstyle{definition}
% \newtheorem{definition}{Definicja}[section]
\newcommand{\cov}{\mathrm{Cov}}
\def\[{\left[}
\def\]{\right]}
\def\real{{\mathord{\mathbb R}}}
\def\N{{\mathord{\mathbb N}}}
\def\Dom{\mathrm{Dom}}
\def\Var{\mathrm{Var}}
% \newcommand{\p}{\mathbb{P}}
\newcommand{\pr}{\mathbb{P}}
\def\P{\mathbb{P}}
% \newcommand{\R}{\mathbb{R}}
%\newcommand{\N}{\mathbb{N}}
\newcommand{\Z}{\mathbb{Z}}
% \newcommand{\C}{\mathbb{C}}
% \newcommand{\E}{\mathbb{E}}

\newenvironment{Proof}{\removelastskip\par\medskip
\noindent{\em Proof.} \rm}{\penalty-20\null\hfill$\square$\par\medbreak}

\newenvironment{Proofx}{\removelastskip\par\medskip
\noindent{\em Proof.} \rm}{\par}

\newenvironment{Proofy}{\removelastskip\par\medskip
\noindent{\em Proof} \rm}{\penalty-20\null\hfill$\square$\par\medbreak}

\allowdisplaybreaks

\numberwithin{equation}{section}

% \usepackage{refcheck}
%%%%%%%%%%%%%%%%%%%%%%%%%%%%%% for drawing pictures by tikz

\usepackage{graphicx}
\usepackage{flushend,cuted}
\usepackage{bm}
\usepackage{tabularx}
%\usepackage{color}
\usepackage{indentfirst}
\usepackage{amssymb}
\usepackage{xparse}
\usepackage{tikz}
\usepackage{mdwlist}
\usepackage{tkz-graph}

\GraphInit[vstyle = Shade]
\usetikzlibrary[intersections,
positioning,
petri,
backgrounds,
fit,
decorations.pathmorphing,
arrows,
arrows.meta,
bending,
calc,
intersections,
through,
backgrounds,
shapes.geometric,
quotes,
matrix,
trees,
shapes.symbols,
graphs,
math,
patterns,
external,
scopes,
matrix,
lindenmayersystems,
shapes.callouts,
shapes.misc,
angles,
shapes.arrows,
shadings]
%%%%%%%%%%%%%%%%%%%%%%%%%%%%%%%%%%%%%%%%

\usetikzlibrary{matrix,calc}

\newcommand{\convexpath}[2]{
[   
    create hullnodes/.code={
        \global\edef\namelist{#1}
        \foreach [count=\counter] \nodename in \namelist {
            \global\edef\numberofnodes{\counter}
            \node at (\nodename) [draw=none,name=hullnode\counter] {};
        }
        \node at (hullnode\numberofnodes) [name=hullnode0,draw=none] {};
        \pgfmathtruncatemacro\lastnumber{\numberofnodes+1}
        \node at (hullnode1) [name=hullnode\lastnumber,draw=none] {};
    },
    create hullnodes
]
($(hullnode1)!#2!-90:(hullnode0)$)
\foreach [
    evaluate=\currentnode as \previousnode using \currentnode-1,
    evaluate=\currentnode as \nextnode using \currentnode+1
    ] \currentnode in {1,...,\numberofnodes} {
-- ($(hullnode\currentnode)!#2!-90:(hullnode\previousnode)$)
  let \p1 = ($(hullnode\currentnode)!#2!-90:(hullnode\previousnode) - (hullnode\currentnode)$),
    \n1 = {atan2(\y1,\x1)},
    \p2 = ($(hullnode\currentnode)!#2!90:(hullnode\nextnode) - (hullnode\currentnode)$),
    \n2 = {atan2(\y2,\x2)},
    \n{delta} = {-Mod(\n1-\n2,360)}
  in 
    {arc [start angle=\n1, delta angle=\n{delta}, radius=#2]}
}
-- cycle
}

\tikzset{hide labels/.style={every label/.append style={text opacity=0}}}

\begin{document}
\title{
\huge
Formulization of $k$-hop count in 1d disk model
} 

%\author{
  %Qingwei Liu\footnote{\href{mailto:qingwei.liu@unumelb.edu.au}{qingwei.liu@unimelb.edu.au}}
  %\qquad
      %Nicolas Privault\footnote{
%\href{mailto:nprivault@ntu.edu.sg}{nprivault@ntu.edu.sg}
%}
  %\\
%\small
%Division of Mathematical Sciences
%\\
%\small
%School of Physical and Mathematical Sciences
%\\
%\small
%Nanyang Technological University
%\\
%\small
%21 Nanyang Link, Singapore 637371
%}

\maketitle
%\section*{First approach: sum of indicators}
For $r>0$, $k\ge2$, let $J_\ell:=[(\ell-1)r,\ell r]$ denote some intervals with equal length, $\ell=1,\dots,k$, called {\em cell}s.
For each $\ell=1,\dots,k-1$, there exists a sequence of i.i.d. random variables $\{X^{(\ell)}_i\}_{i=1}^{m_\ell}$ with a common distribution $\mu_\ell$ such that $\mu_{\ell}(J_\ell)=1$.
\footnote{\textcolor{red}{It is very likely that we need assume $\mu_{\ell}$ to be non-atomic.}}
Denote 
$$\mathcal{P}:=\cup_{\ell=1}^k\left\{X^{(\ell)}_{1},\dots,X^{(\ell)}_{m_\ell}\right\}\cup\{0,y\},$$ 
with $y:=kr-\tau$ for some $\tau\in(0,r)$.


In this article, we consider a random graph (also a random connection model) constructed as follows.
For any two distinct points $u,w\in\mathcal{P}$, if $|u-w|\le r$, then they are connected by a random edge with probability $p\in(0,1)$ independently, denoted by $u\leftrightarrow w$.
Let the resulting random graph be denoted by $\mathbb{G}_p(\mathcal{P})$.

We consider the count $W_y$ of $k$-hop connecting $0$ to $y$, that is a path of order $k+1$, in the random connection model $\mathbb{G}_p(\mathcal{P})$.
More formally, we can write it as 
\begin{equation}
	W_y=\sum_{\alpha_1\in[m_1],\dots,\alpha_{k-1}\in[m_{k-1}]}\bone(0\leftrightarrow X_{\alpha_1}^{(1)})\left(\prod_{\ell=1}^{k-2}\bone(X_{\alpha_\ell}^{(\ell)}\leftrightarrow X_{\alpha_{\ell+1}}^{(\ell+1)})\right)\bone(X_{\alpha_{k-1}}^{(k-1)}\leftrightarrow y).\label{def-khop1}
\end{equation}

%\section*{Second approach: functional form}
Alternatively, we can also represent $\sigma_k$ in a functional form. 
Let $h:\R_+\times\R_+\to[0,1]$ be a connection function defined as 
\begin{equation}
	h(u,w)=p\bone(|u-w|\le r).
\end{equation}
Denote $M:=m_1+m_2+\cdots+m_{k-1}+1$, and 
$$
L_\ell:=m_1+m_2+\cdots+m_{\ell},
$$
for $\ell=1,\dots,k-1$.
% the total number of random points. 
Let $\mathcal{Y}:=\{Y_{i,j}\}_{0\le i,j\le M}$ be a group of i.i.d. random variables uniformly distributed on $[0,1]$, independ of $\cup_{\ell=1}^k\left\{X^{(\ell)}_{1},\dots,X^{(\ell)}_{m_\ell}\right\}$.
\begin{align}\label{def-khop2}
	W_y&=\sum_{\alpha_1\in[m_1],\dots,\alpha_{k-1}\in[m_{k-1}]}f_y(X_{\alpha_1}^{(1)},\dots,X_{\alpha_{k-1}}^{(k-1)},\mathcal{Y}),
\end{align}
where
\begin{align}
	&f_y(x_{\alpha_1}^{(1)},\dots,x_{\alpha_{k-1}}^{(k-1)},\{u_{i,j}\}_{0\le i,j\le M})\nonumber\\
	&:=\bone\left(u_{0,\alpha_1}\le h(0,x_{\alpha_1}^{(1)})\right)\times\bone\left(u_{\alpha_1,L_1+\alpha_2}\le h(x_{\alpha_1}^{(1)},x_{\alpha_2}^{(2)})\right)\times\cdots\nonumber\\
	&\ \ \ \times\bone\left(u_{L_{k-3}+\alpha_{k-2},L_{k-2}+\alpha_{k-1}}\le h(x_{\alpha_{k-2}}^{(k-2)},x_{\alpha_{k-1}}^{(k-1)})\right)\times\bone\left(u_{L_{k-2}+\alpha_{k-1},M}\le h(x_{\alpha_{k-1}}^{(k-1)},y)\right).
\end{align}

For any $x^{(\ell)}\in J_{\ell}$, $\ell=1,\dots,k-1$, 
a key observation is that with $y=kr-\tau$, %to ensure 
\begin{equation*}
	|x^{(1)}|\vee|x^{(2)}-x^{(1)}|\vee\cdots\vee|x^{(k-1)}-x^{(k-2)}|\vee|y-x^{(k-1)}|\le r,
\end{equation*}
is equivalent to
\begin{equation}\label{obs1}
	0\le r-x^{(1)} \le 2r-x^{(2)}\le \cdots \le (k-1)r-x^{(k-1)}\le \tau.
\end{equation}
Combining \eqref{obs1} with \eqref{def-khop1} and \eqref{def-khop2}, we can rewrite $W_y$ as
\begin{equation}
	%W_y=
	\sum_{\alpha_1,\dots,\alpha_{k-1}}\bone\left(r-X^{(1)}_{\alpha_1}\le 2r- X^{(2)}_{\alpha_2}\le\cdots\le (k-1)r-X^{(k-1)}_{\alpha_{k-1}}\le\tau\right)Y_{0,\alpha_1}Y_{\alpha_1,L_1+\alpha_2}\cdots Y_{L_{k-2}+\alpha_{k-1},M}.
\end{equation}
In this form, it is clear that when $\mu_1=\mu_2=\cdots=\mu_{k-1}=\mu$, we can understand $W_y$ as a generalised $U$-statistics.


\begin{remark}
	To ensure a normal approximation, we only need to let some $m_\ell\to\infty$, not necessarily $m_1=\cdots=m_{k-1}=m$ and $m\to\infty$. 
\end{remark}
\begin{remark}
  To ensure a Poisson approximation, there is no need to force $\E(W)$ converges to a constant $\mu$. 
  Poisson approximation by Stein-Chen method does not require $\E(W)$ be a constant. In fact, $\E(W)$ can be large.
\end{remark}
\end{document}
